% \documentclass{report}
\documentclass[a4paper,12pt]{article}
\usepackage{mystyle}
\usepackage{commands}
\mathtoolsset{showonlyrefs=true}

% remember that docmute package neglects all the preambles of the included .tex files. 
\begin{document}
% note that \chapter is not available for article
\subsection{Examples of RKHSs}


\begin{ex} (RKHS generated by a finite dimensional kernel) \label{ex rkhs finite dim kernel}
	Let \( K: E \times E \to \mathbb{F} \) be a kernel and suppose that the inner product space spanned by \( \{K(\cdot ,x)\}_{x \in E} \) is of finite dimensional. Then it is an Hilbert space; call it \( H \). Theorem\ref{Moore Theorem} says that \( H \) is also the RKHS with reproducing kernel \( K \).
	We construct a RKHS in an Euclidean space that is isomorphic to \( H \).
	A basis \( B:= \left< e_1, \ldots, e_d  \right> \) of \( H \), for instance, \( \left< K(\cdot ,x_1), \ldots ,K(\cdot ,x_d) \right> \) for some \( x_i \)'s,
	defines a bijective linear mapping
	\[
		A : H \ni f = \sum_{i=1}^{d} a_i e_i \mapsto (a_i) \in \mathbb{F}^d
	\]
	The inner product of \( H \) is then given by
	\[
		\ip{f}{g}_H
		= \sum_{i,j=1}^{d} a_i \overline{b_j}K(x_j,x_i)
		= \transpose{(Af)}M_K \overline{Ag},
	\]
	where \( M_K:= K_{ij} := \left( K(x_i,x_j) \right) \) is a \( d \times d \) (Hermitian) symmetric matrix.
	
	Now assume that \( M_K \) is positive definite in the matrix sense, that is, assume that every eigenvalue \( \lambda_i \) of \( M_K \) is positive.
	Let \( \left< u_1, \ldots, u_d  \right> \) be an orthonormal basis derived by the spectral decomposition of \( M_K \), for which we have \( M_K = \sum_{i=1} \lambda_i u_i \transpose{\overline{u_i}} \).
	As before, this basis defines a bijective linear mapping
	\[
		B : \mathbb{F}^{d} \ni a = \sum_{i=1}^{d} \alpha_i u_i \mapsto (\alpha_i) \in  \mathbb{F}^{d}.
	\]
	It then follows that
	\[
		\ip{f}{g}_H
		= \sum_{i=1}^{d} \left( \transpose{a}u_i \right) \overline{u_i b}
		= \sum_{i=1}^{d} \lambda_i \alpha_i \overline{\beta_i}
		= \sum_{i=1}^{d} \lambda_i(BAf)_i \overline{(BAg)_i}.
	\]
	Thus, the mapping \( F(f) := BAf \) is an isomorphism between \( H \) and \( \mathbb{F}^d \) equipped with the inner product
	\[
		\ip{a}{b}_K := \transpose{a}M_K^{-1}\overline{b}
		= \sum_{i=1}^{d} \frac{\alpha_i\overline{\beta_i}}{\lambda_i}.
	\]
	Moreover, \( \left( H, \ip{\cdot }{\cdot }_{K} \right) \) is a RKHS with RK \( M_K \). In fact, for \( a \in \mathbb{F}^d \) and \( k_i:= \overline{(K_{i1}, \ldots, K_{in})}^{\top} \), we have
	\[
		\ip{a}{k_i}_K = \transpose{a}M_K^{-1}\overline{k_i} = \transpose{a}M_K^{-1} (M_K)_i = a_i,
	\]
	as required.
	\fin\end{ex}

\begin{ex} (RKHS of polynomial a kernel)
	Since the linear space spanned by a polynomial kernel of degree \( d \) (Example\ref{ex polynomial kernel}) is finite dimensional, Example\ref{ex rkhs finite dim kernel} tells us that the corresponding RKHS coincides with the space consisting of polynomials of degree at most \( d \).
	\fin\end{ex}

\begin{ex} (RKHS of a Gaussian RBF kernel)
	In Corollary\ref{rkhs by Fourier}, take \( r \in L^{1}(\mathbb{R}^d, dt) \) as
	\[
		r(t) :\mathbb{R} ^d \ni t \mapsto \frac{\sigma}{2 \pi} \exp \left( - \frac{\sigma^2}{2} \ip{t}{t} \right) \in \mathbb{R}.
	\]
	The resulting kernel is
	\[
		K(x,y) : \mathbb{R} ^d \ni t \mapsto \exp \left( -\frac{1}{2 \sigma^2} \norm{x-y}^2 \right) \in \mathbb{R},
	\]
	called \textit{Gaussian RBF kernel}. The associated RKHS is given by
	\[
		H = \{f \in L^2(\mathbb{R}^d, dt) \mid \int |\hat{f}(t)|^2 \exp \left( \frac{\sigma^2}{2} \ip{t}{t} \right) dt < \infty \}
	\]
	equipped with the inner product (equivalent to)
	\[
		\ip{f}{g} = \int \hat{f}(t)\cdot \hat{g}(t) \exp \left( \frac{\sigma^2}{2} \ip{t}{t} \right) dt
	\]
	\fin\end{ex}

\begin{ex} (RKHS of a Laplacian kernel)
	
	\fin\end{ex}
\end{document}
