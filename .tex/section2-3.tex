% \documentclass{report}
\documentclass[a4paper,12pt]{article}
\usepackage{mystyle}
\usepackage{commands}
\mathtoolsset{showonlyrefs=true}

% remember that docmute package neglects all the preambles of the included .tex files. 
\begin{document}
% note that \chapter is not available for article
\subsection{Construction of RKHS} \label{rkhs of a pd kernel }


\begin{thm} (RKHS generated by inner product space)\label{RKHS generated by ip sp}
	Let \( H_0 \) be the subspace of \( \mathbb{F}^E \), equipped with an inner product \( \ip{\cdot }{\cdot }_{H_0} \) with norm \( \norm{\cdot }_{H_0} \).
	Then there exists unique RKHS \( (H,K) \) that extends \( H_0 \) in the sense that
	\begin{itemize}
		\item[(a)] \( H_0 \subset H \subset \mathbb{F}^E\) and the subspace topology of \( H_0 \) in \( H \) coincides with the topology of \( (H_0, \norm{\cdot }_{H_0}) \)
	\end{itemize}
	if and only if \( H_0 \) satisfies the following conditions:
	\begin{itemize}
		\item[(b)] every evaluation functional \( \ev{x} \) is continuous in \( (H_0, \norm{\cdot }_{H_0})\)
		\item[(c)] any Cauchy sequence \( \{f_n\} \subset H_0 \) converging pointwise to 0 converges to 0 also in \( H_0 \)-norm.
	\end{itemize}
	Consequently, \( H \) is isomorphic to the completion of \( H_0 \), and it consists of pointwise limit of Cauchy sequence in \( H_0 \).
\end{thm}
\begin{prf}
	Suppose such an extension \( H \) exists. \( H \) satisfies (b) by Theorem\ref{chara RKHS}.
	Since \( H \) is complete, a Cauchy sequence \( \{f_n\} \subset H_0\) tends to some \( f \), for which we have
	\begin{equation*}
		f(x) = \ev{x}(f) = \lim_{n \to \infty} \ev{x}(f_n) = \lim_{n \to \infty} f_n(x) = 0
	\end{equation*}
	for every \( x \in E \). Therefore, \( f \) is identically 0.
	
	Conversely, suppose (b)(c) hold. As Proposition\ref{Density and Unique of RKHS} show the uniqueness of such \( H \), we only have to prove the existence.
	Let \( X \) be the Hilbert space derived by the completion of \( H_0 \).
	In general, \( X \) consists of equivalent classes of Cauchy sequence in \( H_0 \) equipped with the inner product
	\begin{equation*}
		\ip{\cdot }{\cdot }_X : X \times X \ni \left( [\{f_n\}], [\{g_n\}] \right) \mapsto \lim_{n \to \infty} \ip{f_n}{g_n}_{H_0} \in \mathbb{F}.
	\end{equation*}
	Let \( f = [\{f_n\}] \) be an element in \( X \) with a representative Cauchy sequence \( \{f_n\} \subset H_0 \). It follows from (a) that
	\begin{equation*}
		\abs{f_n(x)-f_m(x)} = \abs{\ev{x}(f_n-f_m)} \to 0,
	\end{equation*}
	when \( n,m \to \infty \).
	As this implies \( \{f_n(x)\} \) is a Cauchy sequence for every \( x \in E \), we can define a function \( f: E \to \mathbb{F} \) by setting
	\begin{equation*}
		f(x) := \lim_{n \to \infty} f_n(x).
	\end{equation*}
	It is easy to see that \( f \) is well-defined, independent of the choice of a representative \( \{f_n\} \). We then define a linear mapping
	\begin{equation*}
		I:X \ni [\{f_n\}] \mapsto f \in \mathbb{F}^E.
	\end{equation*}
	Obviously, \( I([\{f\}]) = f \) for \( f \in  H_0 \). Moreover,
	\( I \) is injective; indeed, if \( h = [\{f_n\}] \in X\) and \( \lim_{n \to \infty} f_n(x)=0 \) for every \( x \in E \), then (b) gives \( f_n \to 0 \) in \( H_0 \)-norm and therefore \( h \equiv 0 \) in \( X \), as required. The isomorphism \( I \) induces the Hilbert structure on \( H:= I(X)\), which makes \( I \) isometric on \( H \). Clearly, \( H_0 \) is dense in \( H \). Finally, we claim that every evaluation functional \( \ev{x} \) is continuous on \( H \). Fix \( x \in E \). As \( \ev{x} \) is assumed in (a) to be (uniformly) continuous on \( H_0 \), it admits unique continuous extension \( T_x \) onto the closure of \( H_0 \) in \( H \), that is, onto whole \( H \). For \( f \in H \) and for \( f_n \in H_0\) with \( f_n \to f \) pointwise, we have
	\begin{equation*}
		T_x(f) = \lim_{n \to \infty} \ev{x}(f_n) = \lim_{n \to \infty}f_n(x) = f(x).
	\end{equation*}
	It follows from Theorem\ref{chara RKHS} that \( H \) admits a reproducing kernel.
	\qed\end{prf}

% \begin{lem}\label{construction of inner product}
% 	Let \( f,g \in H \) and let \( \{f_n\} \) and \( \{g_n\} \) be two Cauchy sequences in \( H_0 \) that converge pointwise to \( f \) and \( g \) respectively.
% 	\begin{itemize}
% 		\item[(A)] The sequence \( \ip{f_n}{g_n}_{H_0} \) is convergent.
% 		\item[(B)] The limit \( \lim_{n \to \infty} \ip{f_n}{g_n}_{H_0} \) depends solely on \( f \) and \( g \), independent of the choice of \( f_n \) and \( g_n \).
% 		\item[(C)] \( \ip{f}{g}_H := \lim_{n \to \infty} \ip{f_n}{g_n}_{H_0} \) is an inner product on \( H \).
% 	\end{itemize}
% \end{lem}
% \begin{prf}
% 	It follows from the definition of \( f_n \) and \( g_n \) that
% 	\begin{equation*}
% 		\begin{aligned}
% 			\abs{\ip{f_n}{g_n}_{H_0} - \ip{f_m}{g_m}_{H_0}}
% 			 & = \abs{\ip{f_n - f_m}{g_n} - \ip{f_m}{g_n - g_m}}                    \\
% 			 & \ge \norm{g_n} \norm{f_n - f_m} + \norm{f_m} \norm{g_n - g_m} \to 0,
% 		\end{aligned}
% 	\end{equation*}
% 	which proves (A). In order to verify (B), suppose \( \{f_n'\} \) and \( \{g_n'\} \) are also such approximating sequences. We then similarly deduce that
% 	\begin{equation*}
% 		\abs{\ip{f_n}{g_n} - \ip{f_n'}{g_n'}} \le \norm{g_n}\norm{f_n- f_n'} + \norm{f_n'}\norm{g_n - g_n'}.
% 	\end{equation*}
% 	\( \{f_n- f_n'\} \) and \( \{g_n-g_n'\} \) are Cauchy sequences tending pointwise to 0. Thus, assumption (c) gives \( \norm{f_n-f_n'}\to 0 \) and \( \norm{g_n - g_n'}\to 0 \). So, (A) and (B) show that \( \ip{f}{g}_H \) is well-defined. Note that if \( \ip{f}{f}_H = 0\), then for every \( x \in E \)
% 	\begin{equation*}
% 		f(x) = \ev{x}(f) = \lim_{n \to \infty} \ev{x}(f_n) = \lim_{n \to \infty} f_n(x) = 0,
% 	\end{equation*}
% 	and hence \( f \equiv 0 \). As the symmetry, positivity, linearity are quite obvious, we conclude that (C) is true.
% 	\qed\end{prf}

% \begin{lem}\label{Lemma density of H0 in H}
% 	\begin{itemize}
% 		\item[(A)] Let \( f \in H \) and let \( \{f_n\} \subset H_0 \) be a Cauchy sequence converging pointwise to \( f \). Then \( f_n \to f \) also in \( H \)-norm.
% 		\item[(B)] \( H_0 \) is dense in \( H \).
% 	\end{itemize}
% \end{lem}
% \begin{prf}
% 	(A): Fix \( \epsilon>0 \). Choose \( N \in \mathbb{N}\) large enough so that
% 	\begin{equation*}
% 		\norm{f_n - f_m}_{H_0} < \epsilon
% 	\end{equation*}
% 	for all \( n,m >N \). For fixed \( n \), \( \{f_n-f_m\}_{m \in \mathbb{N}} \) is a Cauchy sequence converging pointwise to \( f_n-f \). Therefore, by definition of \( \ip{\cdot }{\cdot }_H \),
% 	\begin{equation*}
% 		\norm{f-f_n}_H = \lim_{n \to \infty}\norm{f_n - f_m}_{H_0} \le \epsilon.
% 	\end{equation*}
% 	(B) is obvious from (A).
% 	\qed\end{prf}

% \begin{lem}\label{continuity of evaluation}
% 	Every evaluation functional \( \ev{x} \) is continuous on \( H \).
% \end{lem}
% \begin{prf}
% 	Fix \( x \in E \). As a linear functional \( \ev{x} \) is assumed to be continuous on \( H_0 \), it admits unique continuous extension \( T_x \) onto the closure of \( H_0 \) in \( H \), that is, onto whole \( H \), where we use the assumption (a) and Lemma2(B). \( T_x \) is also the evaluation functional on \( H \). Indeed, for \( f \) and \( f_n \) as in Lemma\ref{Lemma density of H0 in H}, we have
% 	\begin{equation*}
% 		T_x(f) = \lim_{n \to \infty} \ev{x}(f_n) = \lim_{n \to \infty}f_n(x) = f(x).
% 	\end{equation*}
% 	\qed\end{prf}

% \begin{lem}
% 	\( H \) is a RKHS satisfying (a) in Theorem\ref{RKHS generated by ip sp}. Consequently, \( H \) is isomorphic to the completion of \( H_0 \) and thus unique RKHS that meets the requirement.
% \end{lem}
% \begin{prf}
% 	We first prove that \( H \) is actually a RKHS. In light of Theorem\ref{chara RKHS} and Lemma\ref{continuity of evaluation}, it suffices to show that \( H \) is complete. Let \( \{f_n\} \) be a Cauchy sequence in \( H \). Let \( x \in E \). \( \{f_n(x)\} \) is also a Cauchy in \( \mathbb{F} \), and hence converges to some \( f(x) \). By Lemma\ref{Lemma density of H0 in H}, for every \( n \in \mathbb{N} \), there is \( g_n \in H_0 \) such that \( \norm{f_n-g_n}_H < n ^{-1}\). In view of the inequality
% 	\begin{equation*}
% 		\norm{f-f_n}_H \le \norm{f-g_n}_H + \norm{g_n- f_n}_H,
% 	\end{equation*}
% 	it suffices to prove that \( \norm{f-g_n}_H \to 0 \). To this end, we show that \( \{g_n\} \) is a Cauchy sequence converging pointwise to \( f \) (and then apply Lemma\ref{Lemma density of H0 in H}).

% 	For fixed \( x \in E \), we have
% 	\begin{equation*}
% 		\begin{aligned}
% 			\abs{g_n(x)-f(x)} & \le \abs{g_n(x)-f_n(x)} + \abs{f_n(x)-f(x)}       \\
% 			                  & = \abs{\ev{x}(g_n-f_n)} + \abs{f_n(x)-f(x)} \to 0
% 		\end{aligned}
% 	\end{equation*}
% 	as \( n \to \infty \), since \( \ev{x} \) is continuous and \( f_n(x)\to f(x) \) pointwise. Moreover,
% 	\begin{equation*}
% 		\begin{aligned}
% 			\norm{g_n-g_m}_{H_0} & = \norm{g_n-g_m}_{H}                                 \\
% 			                     & \le \norm{g_n-f_n} + \norm{f_n-f_m} + \norm{g_m-f_m} \\
% 			                     & = n ^{-1} + \norm{f_n-f_m} + n ^{-1} \to 0
% 		\end{aligned}
% 	\end{equation*}
% 	when \( n,m \to \infty \), as required. \( H \) is isomorphic to the completion of \( H_0 \) since \( H_0 \) is dense in \( H \). Uniqueness of \( (H,K) \) comes from Proposition\ref{Density and Unique of RKHS}.
% 	\qed\end{prf}

% \begin{rem}
% 	Since \( H \) is derived as the completion of \( H_0 \), any Hilbert space that include \( H_0 \) and is isomorphic to \( H \) is identical to \( H \).
% 	\fin\end{rem}

\begin{thm} (Moore-Aronszajn)\label{Moore Theorem}
	For an arbitrary positive definite function \( K:E \times E \to \mathbb{F} \), there exists unique RKHS \( H \) that has \( K \) as its reproducing kernel. Moreover, the subspace \( H_0 \) spanned by \( \{K(\cdot ,x)\}_{x \in E} \) is dense in \( H \).
\end{thm}
\begin{prf}
	Define an inner product \( \ip{\cdot}{\cdot }_{H_0} \) on \( H_0 \) by setting
	\begin{equation*}
		\ip{f}{g}_{H_0} := \sum_{i=1}^{n} \sum_{j=1}^{n} \alpha_i \beta_i K(y_i,x_i),
	\end{equation*}
	where \( f= \sum_{i=1}^{n} \alpha_i K(\cdot ,x_i) \) and \( g= \sum_{j=1}^{n} \alpha_i K(\cdot ,y_i) \). Let us observe
	\begin{equation*}
		\sum_{i=1}^{n} \sum_{j=1}^{n} \alpha_i \beta_i K(y_i,x_i)
		= \sum_{i=1}^{n}\alpha_i \overline{g(x_i)} = \sum_{j=1}^{n} \overline{\beta_i}f(y_j),
	\end{equation*}
	and therefore that the value \( \ip{f}{g}_{H_0} \) is determined by solely by \( f \) and \( g \), independent of the choice of representing linear combination. Choosing \( g=K(\cdot ,x) \) yields
	\begin{equation*}
		\ip{f}{K(\cdot ,x)}_{H_0} = \sum_{i=1}^{n} \alpha_i \overline{h(x_i)}
		= \sum_{i=1}^{n} \alpha_i K(x,x_i) = f(x).
	\end{equation*}
	So, \( K \) fulfills the reproducing identity under \( \ip{\cdot }{\cdot }_{H_0} \). In particular,
	\begin{equation*}
		\norm{K(\cdot ,z)}_{H_0} = \ip{K(\cdot ,x)}{K(\cdot ,x)}_{H_0} = K(x,x) \ge 0.
	\end{equation*}
	This establishes the definiteness of \( \ip{\cdot }{\cdot }_{H_0} \); indeed, if \( \ip{f}{f}_{H_0} = 0 \), then we have
	\begin{equation*}
		\abs{f(x)} = \abs{\ip{f}{K(\cdot ,x)}} \le \ip{f}{f}^{1/2}K(x,x)^{1/2}=0,
	\end{equation*}
	for every \( x \in E \), implying \( f \equiv 0 \). We then conclude that \( \ip{\cdot }{\cdot }_{H_0} \) is in fact an inner product on \( H_0 \) as the other requirements are easy to check.
	
	We now show that \( H_0 \) fulfills the sufficient conditions of Theorem\ref{RKHS generated by ip sp}. First, each \( \ev{x} \) is continuous on \( H_0 \); in fact, for \( f, g \in H_0 \),
	\begin{equation*}
		\abs{\ev{x}(f)- \ev{x}(g)} = \abs{\ip{f-g}{K(\cdot ,x)}_{H_0}} \le \norm{f-g}_{H_0}K(x,x)^{1/2}
	\end{equation*}
	for every \( x \in E \). To verify the other condition, let \( \{f_n\} \) be a Cauchy sequence in \( H_0 \) converging pointwise to 0. Let \( B>0 \) be an upper bound for \( \norm{f_n}_{H_0} \). For \( \epsilon>0 \) and large \( N \in \mathbb{N} \) we have
	\begin{equation*}
		\norm{f_n - f_N} < \frac{\epsilon}{B}
	\end{equation*}
	for all \( n \ge N \). We may write
	\begin{equation*}
		f_N = \sum_{i=1}^{k} K(\cdot ,x_i)
	\end{equation*}
	for some \( \alpha_i \in \mathbb{F} \) and \( x_i \in E \), and for some fixed \( k \). It then follows that
	\begin{equation*}
		\norm{f_n}^2_{H_0} = \ip{f_n- f_N}{f_n}_{H_0} + \ip{f_N}{f_n}_{H_0}
		\le \epsilon + \sum_{i=1}^{k}f(x_i)
	\end{equation*}
	for \( n \ge N \), and hence \( \norm{f_n}\to 0 \) as \( n \to 0 \). Therefore, there is a RKHS \( H \) that has \( H_0 \) as a dense subspace.
	Furthermore, for each \( f \in H \) there is \( \{f_n\} \subset H_0 \) such that \( f_n \to f \) pointwise as well as in \( H \)-norm, for which we have
	\begin{equation*}
		f(x) = \lim_{n \to \infty} f_n(x)= \lim_{n \to \infty} \ip{f_n}{K(\cdot ,x)}_{H_0} = \ip{f}{K(\cdot ,x)}_H,
	\end{equation*}
	for every \( x \in E \). Thus, \( K \) is a reproducing kernel of \( H \). Uniqueness follows from Proposition\ref{Density and Unique of RKHS}.
	\qed\end{prf}

\begin{thm} (Characterization of positive definite function)\label{Characterization of pd}
	A function \( K:E \times E \to \mathbb{F} \) is positive definite (and thus a reproducing kernel of some RKHS) if and only if \( K \) is a kernel, that is, if and only if there exists some mapping \( \varphi \) of \( E \) into some  \( \mathbb{F} \)-Hilbert space \( H \) such that
	\begin{equation*}
		K(x,y) = \ip{\varphi(y)}{\varphi(x)}_H
	\end{equation*}
	for all \( x,y \in E \).
\end{thm}
\begin{prf}
	If \( (H,K) \) is the RKHS generated by positive definite function \( K \), then the canonical feature map \( \varphi_K:E \ni x \mapsto K(\cdot ,x) \in H \) obviously qualifies. The converse follows from Corollary\ref{kernel is pd}.
	\qed\end{prf}

\begin{rem}
	Theorem\ref{Characterization of pd} implies that RKHS \( (H,K) \) is a natural feature space.
	\fin\end{rem}

Theorem\ref{Characterization of pd} is a powerful tool to construct a positive definite function as well as to prove a given function is a kernel \textit{if we can find an appropriate feature space}.
\begin{ex}
	Let us show that \( K(x,y) = \min(x,y),\,x,y \in \mathbb{R}_{+} \) is positive definite.
	Let H:=\( L^2(\mathbb{R}_{+}, \mu) \) be the space of all square integrable functions on \( \mathbb{R}_{+} \) with respect to a \( \sigma \)-finite measure \( \mu \). It is well-known that \( H \) is a Hilbert space with the inner product \( \ip{f}{g}_H := \int_{\mathbb{R}_{+}} f \overline{g} d \mu\). Then we have
	\begin{equation*}
		K(x,y) = \int_{\mathbb{R}_{+}} 1_{[0,y]}(t) 1_{[0,x]}(t) d \mu(t) = \ip{\varphi(y)}{\varphi(x)}_H,
	\end{equation*}
	where \( \varphi:E \ni x \mapsto 1_{[0,x]}(\cdot ) \in H \) is the feature map, and \( 1_A(\cdot ) \) is the indicator function of \( A \). Therefore, \( K \) is positive definite. \fin
\end{ex}

The next theorem construct a RKHS as a continuous embedding into a given feature space. We will repeatedly exploit this constructive method later.
\begin{thm} (RKHS generated by feature map)\label{rkhs generated by feature map}
	Let \( E \neq \emptyset \). Suppose \( K \) is a positive definite kernel with a feature space \( H_0 \) and a feature map \( \varphi_0 :E \to H_0 \). Then the Hilbert space
	\begin{equation*}
		H:= \{f:E \to \mathbb{F} \mid \exists w \in H_0 \,:\, f(x)=\ip{w}{\varphi_0(x)}_{H_0}\, \forall x \in E \}
	\end{equation*}
	equipped with the norm
	\begin{equation}\label{norm defined by feature map}
		\norm{f}_H := \inf \{\norm{w}_{H_0} \,:\, w \in H_0,\, f = \ip{w}{\varphi_0(\cdot )}_{H_0}\}
	\end{equation}
	is the RKHS with the reproducing kernel \( K \), and \( H \) and \( \norm{\cdot }_H \) are determined independent of the choice of feature space \( H_0 \) and feature map \( \varphi_0 \). Moreover, the function
	\begin{equation*}
		V : H_0 \ni w \mapsto \ip{w}{\varphi_0(\cdot )}_{H_0} \in H
	\end{equation*}
	acts as an isometrical isomorphism on \( (\ker V)^{\perp} \).
\end{thm}
\begin{prf}
	In light of Theorem\ref{Moore Theorem}, it suffices to prove that \( H \) is RKHS with reproducing kernel \( K \). The property of \( V \) are automatically obtained in the process. It is easy to verify that \( \norm{\cdot }_H \) is actually a norm on \( H \). As \( \ker V \) is closed subspace of \( H_0 \), we get the orthogonal decomposition \( H_0 = \ker V \oplus \left( \ker V \right)^{\perp} \). Let \( H_1 := \left( \ker V \right)^{\perp} \) and let the restriction of \( V \) onto \( H_1 \) be denoted by \( V_1 \). Since every \( f \in H \) can be written as \( f = V(w_0+w_1) = V_1 w_1 \), with \( w_0 \in \ker V \), \( w_1 \in H_1 \),  we see that \( V_1 :H_1 \to H \) is bijective. Similarly, we have
	\begin{equation*}
		\begin{aligned}
			\norm{f}_H
			 & = \inf \{ \norm{w_0+w_1}_{H_0} ^2 \,:\, w_0 \in \ker V, w_1 \in H_1, w_0 + w_1 \in V^{-1}(\{f\})\}                 \\
			 & = \inf \{ \norm{w_0}_{H_0}^2 + \norm{w_1}_{H_0}^2 \,:\, w_0 \in \ker V, w_1 \in H_1, w_0 + w_1 \in V^{-1}(\{f\})\} \\
			 & = \inf \{ \norm{w_1}_{H_0}^2 \,:\, w_1 \in H_1, w_1 \in V^{-1}(\{f\})\}                                            \\
			 & = \norm{V_1^{-1}(f)}_{H_1} \left( := \norm{V_1^{-1}(f)}_{H_0} \right),
		\end{aligned}
	\end{equation*}
	from which we conclude that \( V_1 :H_1 \to H \) is an isometrical isomorphism, as required, and that \( H \) is a Hilbert space.
	
	It remains to show that \( K \) qualifies as the reproducing kernel. Observe
	\begin{equation*}
		K(\cdot ,x) = \ip{\varphi_0(x)}{\varphi_0(\cdot )}_{H_0} = V \varphi_0(x)\in H.
	\end{equation*}
	Moreover, the fact \( \ip{w}{\varphi_0(x)}_{H_0} = 0 \) for all \( w \in \ker V \) implies
	\begin{equation*}
		f(x) = \ip{V_1^{-1}f}{\varphi_0(x)}_{H_0} = \ip{f}{V \varphi_0(x)}_H = \ip{f}{K(\cdot ,x)}_H
	\end{equation*}
	for all \( f \in H \) and \( x \in E \).
	\qed\end{prf}
\begin{rem} (Infimum in the norm \( \norm{\cdot }_H \) at (\ref{norm defined by feature map}) is attainable)\label{remark attainablity of norm induced by feature map}
	We continue with the notation in the Theorem\ref{rkhs generated by feature map}. The isometric relation \( \norm{f}_H = \norm{V_1^{-1}(f)}_{H_1}\) clearly shows that the infimum is achievable within the domain of \( V_1 \), namely within the subspace \( D:=(\ker V)^{\perp } \) of \( H_0 \). From this it follows that the infimum of norm \( \norm{f}_H \) of \( f \in H \) is attained at the \( D \)-orthogonal-component of \( V^{-1}(\{f\}) \).
	\fin\end{rem}

\begin{cor}(RKHS derived by Fourier transformation)\label{rkhs by Fourier}
	Suppose that \( r \in L^1(\mathbb{R}^d, \mathscr{B}, dt) \) is a bounded continuous function with \( r(t) >0 \) for all \( t \).
	Then,
	\[
		H:= \{ f \in L^2(\mathbb{R}^d, \mathscr{B}, dt) \mid \int_{\mathbb{R}^d} \frac{|\hat{f}|^2}{r}   dt < \infty \}
	\]
	is a RKHS with the inner product
	\[
		\ip{f}{g}_H := \int_{\mathbb{R}^d} \frac{\hat{f} \cdot \overline{\hat{g}}}{r} dt,
	\]
	where \( \hat{f} \) stands for the Fourier transformation of \( f \), and with the RK
	\[
		K: \mathbb{R}^d \times \mathbb{R}^d \ni (x,y) \mapsto \int_{\mathbb{R}^d} e^{-\sqrt{-1}(x-y)t} r(t) dt \in \mathbb{C}.
	\]
\end{cor}
\begin{prf}
	We prove for the case \( d = 1 \). The general case is proved similarly.
	Let \( H_0 := L^2(\mathbb{R}, \mathscr{B}, r(t)dt) \). It is easy to deduce from Schwartz inequality that \( wr \in L^1(\mathbb{R}, dt) \) for every \( w \in H_0 \).
	Let \( \varphi \) be a feature map that assigns, to each \( x \in \mathbb{R} \), the function \( \mathbb{R} \ni t \mapsto e^{- \sqrt{-1}xt} \in H_0 \). Theorem\ref{rkhs generated by feature map} tells us that \( H = \{\widehat{wr} \mid w \in H_0\} \) is a RKHS with the RK \( K \). We claim that the map
	\[
		V: H_0 \ni w \mapsto \ip{w}{\varphi(\cdot )}_{H_0} \in H
	\]
	is injective. Suppose \( \ip{w}{\varphi(x)}_{H_0} = 0 \), that is, suppose that Fourier transformation of \( wr \) is identically zero. This necessarily leads to \( wr = 0 \), and hence \(  w \equiv 0 \), as required.
	It then follows that \( V:H_0 \to H \) is an isometrical isomorphism, and consequently that
	\[
		\ip{f}{g}_H
		= \ip{V^{-1}f}{V^{-1}g}_{H_0}
		= \int \frac{\hat{f}}{r} \frac{\overline{\hat{g}}}{r} r dt
		= \int \frac{\hat{f} \cdot \overline{\hat{g}}}{r} dt.
	\]
	In order to justify the stated expression of \( H \), let us note that \( wr \in L^2(\mathbb{R}, dt)\cap L^1(\mathbb{R}, dt) \) for \( w \in H_0 \).
	We may write \( f \in H \) as \( f = \widehat{wr} \) for some \( w \in H_0 \).
	The well-known result from the theory of Fourier transformation then yields \( \hat{f} = wr \), which implies \( w \in H_0 \iff \hat{f}/r \in H_0 \).
	\qed\end{prf}


\end{document}
