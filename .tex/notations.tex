\documentclass[a4paper,12pt]{article}
\usepackage{mystyle}
\usepackage{commands}

\begin{document}
\section*{Notations}
\addcontentsline{toc}{section}{Notations}

\begin{enumerate}
	\renewcommand{\labelenumi}{\( \diamond \)}
	\item \( \mathbb{N} \): The set of natural numbers excluding zero.
	\item \( \mathbb{R} \): The set of real numbers.
	\item \( \mathbb{F} \): A scalar field, either \( \mathbb{R} \) or \( \mathbb{C} \).
	\item \( E \): A nonempty set on which typical functions are defined.
	\item \( \overline{c} \): The complex conjugate of a scalar \( c \).
	\item \( \Re s \), \( \Im s \): The real and imaginary part of a scalar \( s \).
	\item \( \ip{\cdot }{\cdot }_{H} \): The inner product in the inner product space \( H \).
	\item \( \ker f \): The kernel or null space of \( f \), i.e., the set of all points at which \( f \) is zero.
	\item \( \ev{x} \): The evaluation function at \( x \), i.e., the linear functional such that \( \ev{x}(f) = f(x) \).
	\item \( \adj{H} \), \( \adj{f} \): The dual space of \( H \) and the adjoint of \( f \).
	\item \( S^{\perp } \): The orthogonal complement of a subspace \( S \).
	\item \( A \perp B\): Every function in \( A \) is orthogonal to that in \( B \).
	\item \( L^p(\Omega, \mathscr{B}, \mu) \): \( L^{p} \) space over a measure space \( (\Omega, \mathscr{B}, \mu) \).
\end{enumerate}

\end{document}