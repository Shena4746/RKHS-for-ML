% \documentclass{report}
\documentclass[a4paper,12pt]{article}
\usepackage{mystyle}
\usepackage{commands}
\mathtoolsset{showonlyrefs=true}

% remember that docmute package neglects all the preambles of the included .tex files. 
\begin{document}
\subsection{Reconstruction via Sum and Product}

\begin{thm} (Sum of RKHSs)
	Let \( (H_1,K_1) \) and \( (H_2,K_2) \) be two \( \mathbb{F} \)-RKHSs of functions on the common set \( E \).
	Then \( K:= K_1 + K_2 \) is the RK of
	\begin{equation*}
		H:= H_1 \oplus H_2:= \{f_1 + f_2 \mid f_1 \in H_1, \, f_2 \in H_2\}
	\end{equation*}
	with the norm
	\begin{equation*}
		\norm{f}_H := \min \{\norm{f_1}_{H_1}+\norm{f_2}_{H_2} \,:\, f = f_1 + f_2,\,f_1 \in H_1,\,f_2 \in H_2\}.
	\end{equation*}
\end{thm}
\begin{prf}
	Let \( F \) be the Hilbert sum of \( H_1 \) and \( H_2 \):
	\begin{equation*}
		F := \{(f_1, f_2)\mid f_1 \in H_1,\, f_2 \in H_2\}
	\end{equation*}
	equipped with an inner product
	\begin{equation*}
		\ip{f}{g}_F := \ip{f_1}{g_1}_{H_1} + \ip{f_2}{g_2}_{H_2}.
	\end{equation*}
	It is easy to see that the map
	\begin{equation*}
		\varphi : E \ni x \mapsto \left( K_1(\cdot ,x), K_2(\cdot ,x) \right) \in F
	\end{equation*}
	is a feature map of \( K \), and that we have
	\begin{equation*}
		\ip{f}{\varphi(x)}_F = f_1(x)+f_2(x)
	\end{equation*}
	for all \( f = (f_1,f_2) \in F \) and \( x \in E \).
	Thus, \( (H,K) \) is a RKHS by Theorem\ref{rkhs generated by feature map}. For attainability of \( \norm{\cdot}_H \), see Remark\ref{remark attainablity of norm induced by feature map}.
	\qed\end{prf}

\begin{thm} (Tensor product of RKHSs)
	Let \( K_1 \) and \( K_2 \) be \( \mathbb{F} \)-kernels defined on \( E_1 \) and \( E_2 \), respectively, and let \( H_1 \) and \( H_2 \) be the corresponding \( \mathbb{F} \)-RKHSs. Set \( H:=H_1 \otimes H_2 \).
	\begin{itemize}
		\item[(a)] Define the product kernel \( K \) of \( K_1 \) and \( K_2 \) via
		      \begin{equation*}
			      K:(E_1 \times E_2) \times (E_1 \times E_2) \ni \left( (x_1, x_2), (y_1,y_2) \right) \mapsto K_1(x_1,y_1)K_2(x_2,y_2) \in \mathbb{F}.
		      \end{equation*}
		      Then \( (H,K) \) is a RKHS.
		\item[(b)] Assume \( (E:=) E_1 = E_2 \). The RKHS the kernel \( K_E(x,y) :=K_1(x,y)K_2(x,y) \) coincides with \( H_E:=\{f|_{E \times E} \mid f \in H_1 \otimes H_2\} \).
	\end{itemize}
\end{thm}
\begin{prf}
	(a) It is easy to see that
	\begin{equation*}
		\varphi:E_1 \times E_2 \ni (x_1,x_2) \mapsto \left( K_1(\cdot ,x_1)K_2(\cdot ,x_2) \right)\in H
	\end{equation*}
	is a feature map of \( K \). Let \( H' \) be the RKHS generated by \( \varphi \) (and hence by \( K \)). Let \( H_0 \) be the subspace of \( H' \) spanned by \( \{K(\cdot ,x)\}_{x \in E_1 \times E_2} \).
	Note that \( H_0 \subset H_1 \bullet H_2\) is dense in \( H' \), and that \( H' \) is isomorphic to the completion of \( H_0 \) and hence to that of \( H_1 \bullet H_2 \). It thus follows that \( H' \) and \( H_1 \otimes H_2 \) must coincide.
	\qed\end{prf}
\end{document}
