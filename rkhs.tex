\documentclass[a4paper,12pt]{article}
% Math Packages
\usepackage{amsmath,amsfonts}
\usepackage{mathrsfs}
\usepackage{amssymb}
\usepackage{amsthm}

% ---------- Environment
\newtheorem{thm}{Theorem}[section]
\newtheorem*{thm*}{Theorem}

\newtheorem{prp}[thm]{Proposition}
\newtheorem*{prp*}{Proposition}

\newtheorem{cor}[thm]{Corollary}
\newtheorem*{cor*}{Corollary}

\newtheorem{lem}[thm]{Lemma}
\newtheorem*{lem*}{Lemma}

\newtheorem{dfn}[thm]{Definition}
\newtheorem*{dfn*}{Definition}

\theoremstyle{remark}
\newtheorem*{prf}{Proof}

\theoremstyle{definition}
\newtheorem*{rem*}{Remark}
\newtheorem{rem}[thm]{Remark}

\theoremstyle{definition}
\newtheorem{ex}[thm]{Example}
\newtheorem*{ex*}{Example}

\theoremstyle{definition}
\newtheorem{exe}[thm]{Exercise}
% --------- Environment

% --------- macros
\newcommand{\ip}[2]{\left<#1, #2 \right>}
\newcommand{\abs}[1]{\left| #1 \right|}
\newcommand{\norm}[1]{\left\| #1 \right\|}
\newcommand{\ev}[1]{\mathrm{ev}_{#1}}


\begin{document}
\section{Reproducing Kernel Hilbert Space}
\begin{dfn}[Reproducing Kernel]
	Let \( E \) be a nonempty set. A function \( K \) defined by
	\[
		K:E \times E \ni (x,y) \mapsto K(x,y) \in \mathbb{F}
	\]
	is called a reproducing kernel of a Hilbert space \( H \) of functions on \( E \) if it satisfies the following conditions:
	\begin{itemize}
		\item[(a)] \( K(\cdot, x) \in H \) for every \( x \in E \)
		\item[(b)]  \( \ip{f}{K(\cdot, x)}_H = f(x) \) for every \( x \in E \) and every \( f \in H \).
	\end{itemize}
	Such Hilbert space is called a reproducing kernel hilbert space (RKHS, for short), and is denoted by \( (H(E),K) \) or \( (H,K) \).
\end{dfn}

\begin{thm}[Characterization of RKHS]\label{chara RKHS}
	A Hilbert space \( H \) of functions on a nonempty set \( E \) admits a reproducing kernel \( K \) if and only if all evaluation functionals \( \{\ev{x} \} _{x \in E} \) are continuous on \( H \).
\end{thm}
\begin{prf}
	Suppose \( (H,K) \) is a RKHS.
	For \( x \in E \) and for \( f \in H \) we have
	\[
		\abs{\ev{x}(f)} = \abs{f(x)} = \abs{ \ip{f}{K(\cdot , x)}} \le \norm{f} \norm{K(\cdot,x )} \le \norm{f} K(x,x) ^{1/2} \to 0
	\]
	as \(  \norm{f}\to 0 \). Thus, \( \ev{x} \) is continuous linear functional (with norm \( K(x,x) ^{1/2} \)).

	Conversely, if \( \ev{x}:H \ni f \mapsto f(x) \in \mathbb{F} \) is continuous, then, by Riesz's representation theorem, there exists \( r_x \in H\) such that
	\begin{equation*}
		\ip{f}{r_x} = f(x)
	\end{equation*}
	for every \( f \in H \). If this happens for every \( x \in E \), then \( K(x,y):=r_x(y) \) is a reproducing kernel of \( H \).
	\qed\end{prf}

\begin{cor}
	Every convergent sequence in RKHS converges pointwise to the same limit.
\end{cor}
\begin{prf}
	\( \abs{ f_n(x) - f(x)} = \abs{\ev{x}(f_n - f)} \to 0 \) when \( f_n \to f \) in norm by continuity of evaluation functional.
	\qed\end{prf}
\begin{dfn}(Positive definite function)
	Let \( E \) be a nonempty set.
	A function \( K: E \times E \to \mathbb{C} \) is called positive definite if for any \( n \in \mathbb{N} \) and for any \( a \in \mathbb{C}^n \) and \( x \in E^n \) there holds
	\begin{equation*}
		\sum_{i=1}^{n} \sum_{j=1}^{n} a_i \overline{a_j}K(x_i,x_j) \ge 0,
	\end{equation*}
	where \( \overline{c} \) is the complex conjugate of \( c \).
\end{dfn}
\begin{prp}
	Suppose \( \varphi \) is a mapping of a set \( E \) into a Hilbert space \( H \). Then the mapping \( K: E \times E \ni (x,y) \mapsto \ip{\varphi(x)}{\varphi(y)} \in \mathbb{C} \) is positive definite.
\end{prp}
\begin{prf}
	For \( a \) and \( x \) taken as in the definition, we have
	\begin{equation*}
		\sum_{i=1}^{n} \sum_{j=1}^{n} a_i \overline{a_j}K(x_i,x_j)
		= \sum_{i=1}^{n} \sum_{j=1}^{n} a_i \overline{a_j} \ip{\varphi(x_i)}{\varphi(x_j)}
		= \norm{\sum_{i=1}^{n} a_i \varphi(x_i) }^2 \ge 0.
	\end{equation*}
	\qed\end{prf}
\begin{prp}
	Every reproducing kernel is positive definite.
\end{prp}
\begin{prf}
	\( \sum_{i=1}^{n} \sum_{j=1}^{n} a_i \overline{a_j}K(x_i,x_j) = \norm{\sum_{i=1}^{n} K(\cdot ,x_i)}^2 \).
	\qed\end{prf}

\begin{prp}
	Every positive definite function \( K:E \times E \to \mathbb{C} \) satisfies
	\begin{itemize}
		\item[(a)] \( K(x,x) \ge 0 \) for every \( x \in E \)
		\item[(b)] \( K(x,y)=\overline{K(y,x)} \) for every \( x,y \in E \)
		\item[(c)] \( \overline{K} \) is also positive definite
		\item[(d)] \( \abs{K(x,y)} \le K(x,x)K(y,y) \) for every \( x,y \in E \).
	\end{itemize}
\end{prp}
\begin{prf}
	(a) and (c) clearly hold. For \( \alpha, \beta \in \mathbb{C} \) and \( x,y \in E \), we have
	\begin{equation*}
		g(\alpha,\beta) := \abs{\alpha}^2 K(x,x) + \alpha \overline{\beta}K(x,y) + \overline{\alpha}\beta K(y,x) + \abs{\beta}^2 K(y,y) \ge 0.
	\end{equation*}
	Choose \( \alpha=\beta=1 \) and \( \alpha = i \), \( \beta=1 \) to get
	\begin{equation*}
		\begin{aligned}
			K(x,y) + K(y,x) = g(1,1) - K(x,x) - K(y,y)     & =: A \in \mathbb{R}  \\
			i K(x,y) - i K(y,x) = g(i,1) - K(x,x) - K(y,y) & =: B \in \mathbb{R}.
		\end{aligned}
	\end{equation*}
	Therefore,
	\begin{equation*}
		\begin{aligned}
			2 K(y,x) & = A + iB  \\
			2 K(x,y) & = A - iB,
		\end{aligned}
	\end{equation*}
	which proves (b). Finally, for \( x,y \in E \) with \( K(x,y) \neq 0 \) and for \( r \in \mathbb{R} \), (b) gives
	\begin{equation*}
		0 \ge g(r,K(x,y)) = r ^2 K(x,x) + 2r \abs{K(x,y)}^2 + \abs{K(x,y)}^2K(y,y).
	\end{equation*}
	As RHS is quadratic in \( r \), it must satisfy
	\begin{equation*}
		\abs{K(x,x)}^4 - \abs{K(x,y)}^2K(x,x)K(y,y) \le 0,
	\end{equation*}
	from which (c) follows.
	\qed\end{prf}

\begin{thm}
	Let \( H_0 \) be the subspace of \( \mathbb{F}^E \), equipped with an inner product \( \ip{\cdot }{\cdot }_{H_0} \) with norm \( \norm{\cdot }_{H_0} \).
	Then there exists unique RKHS \( (H,K) \) that extends \( H_0 \) in the sense that
	\begin{itemize}
		\item[(a)] \( H_0 \subset H \subset \mathbb{F}^E\) and the subspace topology of \( H_0 \) in \( H \) coincides with the topology of \( (H_0, \norm{\cdot }_{H_0}) \)
	\end{itemize}
	if and only if \( H_0 \) satisfies the following conditions:
	\begin{itemize}
		\item[(b)] every evaluation functional \( \ev{x} \) is continuous in \( (H_0, \norm{\cdot }_{H_0})\)
		\item[(c)] any Cauchy sequence \( \{f_n\} \subset H_0 \) converging pointwise to 0 converges to 0 also in \( H_0 \)-norm.
	\end{itemize}
\end{thm}
\begin{prf}
	Suppose such an extension \( H \) exists. \( H \) satisfies (b) by Theorem\ref{chara RKHS}.
	Since \( H \) is complete, Cauchy sequence \( \{f_n\} \subset H_0\) tends to some \( f \), for which we have
	\begin{equation*}
		f(x) = \ev{x}(f) = \lim_{n \to \infty} \ev{x}(f_n) = \lim_{n \to \infty} f_n(x) = 0
	\end{equation*}
	for every \( x \in E \). Therefore, \( f \) is identically 0.

	Conversely, suppose (b)(c) hold. Let \( H \) be the set of all functions \( f \in \mathbb{F}^E \) for which there exists a Cauchy sequence \( \{f_n\} \subset H_0 \) converging pointwise to \( f \). Clearly, \( H_0 \subset H \subset \mathbb{F}^E \). The rest of proof consists of the following Lemmas.
	\qed\end{prf}

\begin{lem}\label{construction of inner product}
	Let \( f,g \in H \) and let \( \{f_n\} \) and \( \{g_n\} \) be two Cauchy sequences in \( H_0 \) that converge pointwise to \( f \) and \( g \) respectively.
	\begin{itemize}
		\item[(A)] The sequence \( \ip{f_n}{g_n}_{H_0} \) is convergent.
		\item[(B)] The limit \( \lim_{n \to \infty} \ip{f_n}{g_n}_{H_0} \) depends solely on \( f \) and \( g \), independent of the choice of \( f_n \) and \( g_n \).
		\item[(C)] \( \ip{f}{g}_H := \lim_{n \to \infty} \ip{f_n}{g_n}_{H_0} \) is an inner product on \( H \).
	\end{itemize}
\end{lem}
\begin{prf}
	It follows from the definition of \( f_n \) and \( g_n \) that
	\begin{equation*}
		\begin{aligned}
			\abs{\ip{f_n}{g_n}_{H_0} - \ip{f_m}{g_m}_{H_0}}
			 & = \abs{\ip{f_n - f_m}{g_n} - \ip{f_m}{g_n - g_m}}                    \\
			 & \ge \norm{g_n} \norm{f_n - f_m} + \norm{f_m} \norm{g_n - g_m} \to 0,
		\end{aligned}
	\end{equation*}
	which proves (A). In order to verify (B), suppose \( \{f_n'\} \) and \( \{g_n'\} \) are also such approximating sequences. We then similarly deduce that
	\begin{equation*}
		\abs{\ip{f_n}{g_n} - \ip{f_n'}{g_n'}} \le \norm{g_n}\norm{f_n- f_n'} + \norm{f_n'}\norm{g_n - g_n'}.
	\end{equation*}
	\( \{f_n- f_n'\} \) and \( \{g_n-g_n'\} \) are Cauchy sequences tending pointwise to 0. Thus, assumption (c) gives \( \norm{f_n-f_n'}\to 0 \) and \( \norm{g_n - g_n'}\to 0 \). So, (A) and (B) show that \( \ip{f}{g}_H \) is well-defined. Note that if \( \ip{f}{f}_H = 0\), then for every \( x \in E \)
	\begin{equation*}
		f(x) = \ev{x}(f) = \lim_{n \to \infty} \ev{x}(f_n) = \lim_{n \to \infty} f_n(x) = 0,
	\end{equation*}
	and hence \( f \equiv 0 \). As the symmetry, positivity, linearity are quite obvious, we conclude that (C) is true.
	\qed\end{prf}

\begin{lem}\label{density of H0 in H}
	\begin{itemize}
		\item[(A)] Let \( f \in H \) and let \( \{f_n\} \subset H_0 \) be a Cauchy sequence converging pointwise to \( f \). Then \( f_n \to f \) also in \( H \)-norm.
		\item[(B)] \( H_0 \) is dense in \( H \).
	\end{itemize}
\end{lem}
\begin{prf}
	(A): Fix \( \epsilon>0 \). Choose \( N \in \mathbb{N}\) large enough so that
	\begin{equation*}
		\norm{f_n - f_m}_{H_0} < \epsilon
	\end{equation*}
	for all \( n,m >N \). For fixed \( n \), \( \{f_n-f_m\}_{m \in \mathbb{N}} \) is a Cauchy sequence converging pointwise to \( f_n-f \). Therefore, by definition of \( \ip{\cdot }{\cdot }_H \),
	\begin{equation*}
		\norm{f-f_n}_H = \lim_{n \to \infty}\norm{f_n - f_m}_{H_0} \le \epsilon.
	\end{equation*}
	(B) is obvious from (A).
	\qed\end{prf}
\end{document}